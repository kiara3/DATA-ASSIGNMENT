\documentclass[14pt]{article}

\usepackage{graphicx}

\begin{document}


\title{AN ELECTRONIC REPORT FOR ROOM ALLOCATION IN MARYSTUART HALL}

\author{MBABAZI SARAH }

\date {\today}

\maketitle

\tableofcontents


\section{ABSTRACT}

Room allocation in MsH is according to levels that are students are allocated basing on their academic level of studying. This process has affected many students including the ministers within the hall. Therefore they should change the system by using the process of first come first serve so as to also give opportunity to other students who always feel isolated and disrespected.\par


\section{INTRODUCTION}

\subsection{Purpose}
This report aims at introducing students in hall of Marystuart to the new and brilliant system of room allocation. This explains the student allocation to different rooms in the hall, the limitations, functionalities and the work plan followed during the process of research as i interviewed a few persons.\par 

This report will also provide the reader with a clear definition on what is being done, why and how it will be done.\par
 

\subsection{Student challenges}
•	Most of the students are isolated and by using this slogan of freshers, they feel out of place and end up leaving in the world of they don’t like us because we are fist years. The disabled are also not recognised in the society thus some may end up getting more sick.\par

•	The ministers of the hall such as cheer leader, senior woman among others are also mixed with the ordinary students not in the same room but on the same level.\par

\subsection{Project goals}

•	Gender balance has to be practiced among the students. First years shouldn’t be put on the lower floor just because they are freshers and the ministers should also be given at least one floor to provide them with their privacy since they are the leaders of the hall.\par

•	More to that, the disabled like the lame, blind, dump. deaf should be always given their special level most especially the first one and each given someone to always be helping her in case of anything, which will make their life easy and they will also feel loved.  \par

\subsection{Definitions, acronyms and abbreviations}
•	i.e.: that is\par
•	e.g.: for example\par
•	MsH: Marystuart Hall\par
•	Mins: Ministers\par

\section{PROJECT RESULTS}

\subsection{Product functionality and screenshots}
Screenshots required.

\begin{figure}[h!]
\includegraphics[width=100mm,scale=0.5]{1.jpg}
\caption{app engine.}
\label{figure1}
\end{figure}

\begin{figure}[h!]
\includegraphics[width=100mm,scale=0.5]{3.jpg}
\caption{odk form.}
\label{figure2}
\end{figure}




\section{LIMITATIONS AND NEXT STEPS}

\subsection{Limitations}
•	The space for the rooms is not enough whereby one room can be put to accommodate a number of students not fit to be in e.g. three, four and sometimes can reach five or six.\par
	
•	Students take long to apply for new room at the end of every semester which makes it hard for the custodians to allocate different rooms to them.\par

•	Some of the students i.e. the first years do late registration thus in the course of doing that, they end up on the lower level.\par



\subsection{Solutions}
•	The number of students allowed to be in room should not exit three i.e. the maximum should be three.\par

•	All the students must pick their application forms one month before the beginning of the exams so as to allow room distribution.\par

•	First year students should be oriented about the disadvantages of doing late registration at their halls of residence so that they should also be given rooms at different levels not only at the first floor.\par

•	Mins should also be recognised by giving them special rooms on one full level.\par


\subsection{Next steps}

Registration will always be done online to allow well distribution of rooms to students by  the custodians.\par

\section{PROCEDURES}
These are the procedures i followed;

•	I interviewed different students in their respective rooms.\par 
•	I collected my data electronically which also involved Google maps to find the location.\par
•	Then i connected my phone to the server by inserting the URL of my app engine and this enabled me to submit the collected data.\par

 

\section{CONCLUSION}
The interview among different students was interesting because in the process of doing my project i got a chance to make new friends and knowing what happens in MsH. I never faced any challange because the interviewees were all friendly.\par

\begin{thebibliography}{10}

\bibitem{latexGuide} The Campusers

Available at \texttt{The community around Makerere University }

\bibitem{latexGuide} The hostels

Available at \texttt{Areas around Wandegeya }



\end{thebibliography}

\end{document}
